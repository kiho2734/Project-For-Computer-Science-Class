\documentclass[12pt]{article}

\pagestyle{empty}
\setcounter{secnumdepth}{0}

\topmargin=0cm
\oddsidemargin=0cm
\textheight=22.0cm
\textwidth=16cm
\parindent=0cm
\parskip=0.15cm
\topskip=0truecm
\raggedbottom
\abovedisplayskip=3mm
\belowdisplayskip=3mm
\abovedisplayshortskip=0mm
\belowdisplayshortskip=2mm
\normalbaselineskip=12pt
\normalbaselines

\begin{document}

\vspace*{0.2in}
\centerline{\bf\Large Diary}

\vspace*{0.2in}
\centerline{\bf\Large Name: Ki Ho Lee   Student ID: 40073402}

\vspace*{0.2in}
\centerline{\bf\Large Team PK-B}

\vspace*{0.2in}
\centerline{\bf\Large 10 January 2019}

\section{Iteration 1}

{\bf Date:} 9 January 2020\\
{\bf Start Time:} 20:15\\
{\bf End Time:} 20:20 \\
{\bf Who:} Pavel Balan, Ivan Garzon, Ki Ho Lee, Nian Liu, Anthony Menard-Gill, Michael Naccache, Ashish Saha, Myriam Tayah, Shuo Zhang\\
{\bf Where:} H-920 \\
{\bf Activities:} Get together and introduce each other.\\
{\bf Outcomes:} Decide to use Slack, Trello and Github and who is going to make it.\\

{\bf Date:} 15 January 2020\\
{\bf Start Time:} 19:15\\
{\bf End Time:} 20:45 \\
{\bf Who:} Pavel Balan, Ivan Garzon, Ki Ho Lee, Nian Liu, Anthony Menard-Gill, Michael Naccache, Ashish Saha, Myriam Tayah, Mona Shayvard\\
{\bf Where:} H-831 \\
{\bf Activities:} Divide the team members into 3 parts which are organisers, documenter and coders. The members who are familiar with JUnit or LaTex volunteered to teach those to the members who have not used them before. Also, discussed where we are going to start.\\
{\bf Outcomes:} Pavel Balan is assigned the organiser. Ki Ho Lee, Nian Liu and Myriam Tayah are assigned the coders. Ivan Garzon, Anthony Menard-Gill, Michael Naccache, Ashish Saha and Mona Shayvard are assigned the documentors. The coders make a meeting on Thursday from 2:30 to 3:30 to learn JUnit and discuss the approach to code for the project.\\

{\bf Date:} 16 January 2020\\
{\bf Start Time:} 2:30\\
{\bf End Time:} 3:30 \\
{\bf Who:} Ki Ho Lee, Nian Liu, Myriam Tayah, Michael Naccache\\
{\bf Where:} H-817 \\
{\bf Activities:} Myraim Tayah shows how to use JUnit. The coders discuss how to make the grid in Java.\\
{\bf Outcomes:} The coders decide to use 2-D array for the grids of Kakuro for start. Also, the coders need to get used to Java GUI, so decide to review it before the next meeting on Monday at 2:30.\\

{\bf Date:} xxx\\
{\bf Start Time:} xxxx\\
{\bf End Time:} xxxx \\
{\bf Who:} xxx,xx,xxx (fullnames)\\
{\bf Where:} xxxx \\
{\bf Activities:} xxxxxxx\\
{\bf Outcomes:} xxxxx\\
%\section{Iteration 2}

%\section{Iteration 3}

\end{document}
