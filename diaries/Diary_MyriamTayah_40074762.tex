\documentclass[12pt]{article}

\pagestyle{empty}
\setcounter{secnumdepth}{0}

\topmargin=0cm
\oddsidemargin=0cm
\textheight=22.0cm
\textwidth=16cm
\parindent=0cm
\parskip=0.15cm
\topskip=0truecm
\raggedbottom
\abovedisplayskip=3mm
\belowdisplayskip=3mm
\abovedisplayshortskip=0mm
\belowdisplayshortskip=2mm
\normalbaselineskip=12pt
\normalbaselines

\begin{document}

\vspace*{0.2in}
\centerline{\bf\Large Diary}

\vspace*{0.2in}
\centerline{\bf\Large Name: Myriam TAYAH  Student ID: 40074762}

\vspace*{0.2in}
\centerline{\bf\Large Team PK-B}

\vspace*{0.2in}
\centerline{\bf\Large 10 January 2019}

\section{Iteration 1}

{\bf Date:} January 9th 2020\\
{\bf Start Time:} 8:00 pm\\
{\bf End Time:} 8:15 pm \\
{\bf Who:} Pavel Balan, Yushun Cheng, Ivan Garzon, Ki Ho Lee, Nian Liu, Anthony Menard-Gill, Michael Naccache, Ashish Saha, Myriam Tayah, Shuo Zhang\\
{\bf Where:} Outside of H-920 \\
{\bf Activities:} At the end of class, Professor Butler allowed us to end class early in order to meet up with our teams and become familiar with our teammates. We also discussed our expectations for the class.\\
{\bf Outcomes:} We exchanged information and got to know each other by face. We also found out that Pavel was going to set up a Slack for the team, which he did right after the meeting. \\

{\bf Date:} January 15th 2020\\
{\bf Start Time:} 7:00\\
{\bf End Time:} 8:55 \\
{\bf Who:} Pavel Balan, Ivan Garzon, Ki Ho Lee, Nian Liu, Anthony Menard-Gill, Michael Naccache, Ashish Saha, Myriam Tayah, Mona Shayvard\\
{\bf Where:} H-831 \\
{\bf Activities:} Before the lab, I met up with Mona for the first time since she was new to the team - she asked me if I was in PK-B and we began to discuss the course and what she had missed. During this first lab, we were sitting in team. We discussed the different roles in a development team, the software development life cycle and how it is tightly related to our project. We talked about AGILE development and the lab instructor gave us time to work as a team while considering the concepts that had been discussed. \\
{\bf Outcomes:} Before the lab, Mona introduced me a little bit to how LaTex works. For the actualy lab, we first began by discussing our experience and what we feel most comfortable doing. After, we separated our roles for the first iteration such that Pavel would be the only organizer, Ashish, Michael, Anthony, Ivan and Mona would be documenters as there would be a lot of documenting to do for this first iteration and, finally, Ki Ho, Nian and I would be the coders. Also, we agreed that using Trello would help us keep up with our tasks and that Ivan would set up our team board. Finally, we said that I would hold a small JUnit tutorial the next day in the afternoon and that Ash would be our reference for JavaFX. After the lab, the other coders and I stayed for around 10 minutes to discuss how we should approach the code and what tools we could use to help us. We agreed on a time for the JUnit tutorial and shared the details to the rest of the team in case anyone else was interested to join us. \\

{\bf Date:} January 16th 2020\\
{\bf Start Time:} 10:00 am\\
{\bf End Time:} 11:30 am \\
{\bf Who:} Myriam Tayah\\
{\bf Where:} Faubourg Building \\
{\bf Activities:} I prepared a list of concepts I want to introduce the other coders to when it comes to unit testing and JUnit. I also looked for online resources, like already-written code that I could use to simplify my tutorial and practiced.\\
{\bf Outcomes:} I was ready for the tutorial taking place later that day.\\

{\bf Date:} January 16th 2020\\
{\bf Start Time:} 2:00 pm\\
{\bf End Time:} 3:40 pm \\
{\bf Who:} Ki Ho Lee, Nian Liu, Michael Naccache, Myriam Tayah\\
{\bf Where:} H-831 \\
{\bf Activities:} I introduced the others to a few concepts of unit testing with JUnit. I also explained the difference between unit tests and integration tests. I used a Calculator class I found online and we played around with it. Most importantly, we looked at how to set up unit tests with JUnit in Eclipse and a few industry standards. I did the tutorial twice because Michael joined us a bit after our first try. I asked them a few questions to make sure they were comfortable with everything. We also spent a little bit of time discussing our impressions with the project and got to further know each other at the end. \\
{\bf Outcomes:} I feel as if everyone was at least a bit more comfortable with it. I made sure to ask them questions to make sure that it was clear. Ki Ho, Nian and I also decided to meet up on monday to discuss our progress when it comes to learning about JavaFX.\\

{\bf Date:} January 18th 2020\\
{\bf Start Time:} 5:00 pm\\
{\bf End Time:} 6:35 pm \\
{\bf Who:} Myriam Tayah \\
{\bf Where:} At home \\
{\bf Activities:} I watched quite a few tutorials on YouTube about JavaFX. I also read an article about the differences between JavaFX and Swing.\\
{\bf Outcomes:} I realized that in the context of this project, there would be no difference between the two libraries except that JavaFX could be easier to visualize.\\

{\bf Date:} January 19th 2020\\
{\bf Start Time:} 7:40 pm\\
{\bf End Time:} 10:15 pm \\
{\bf Who:} Myriam Tayah \\
{\bf Where:} At home \\
{\bf Activities:} I tried to set up JavaFX on my laptop which took me quite a long time. I did not figure it out completely. I realized I did not send my information to Pavel to become a contributor in the Git Repository. \\
{\bf Outcomes:} I think that I did not get JavaFX set up properly.\\

{\bf Date:} January 20th 2020\\
{\bf Start Time:} 2:45 pm\\
{\bf End Time:} 3:55 pm \\
{\bf Who:} Ki Ho Lee, Nian Liu, Ashish Saha, Myriam Tayah\\
{\bf Where:} H-841 \\
{\bf Activities:} I arrived a little bit in advance and set up a Trello board for the coders of the first iteration (columns: to do, ongoing, done, blocked). When Nian and Ki Ho arrived, we talked about the research we had made and how we could separate the tasks for each use-case. As a good surprise, Pavel came to say hi and surprised us with Timbits. Also, over the weekend, we had discussed about our meeting and Ashish reached out saying he could help us with JavaFX and he came and gave us tutorial about it, especially for its set up. We agreed that we should use JavaFX because it would be easier for the coders of the next iterations to continue with the code we would have made available to them. \\
{\bf Outcomes:} Our tasks have been separated and assigned in the Trello board. We know more about JavaFX. It is nice to see that everyone in the team is willing to help the others and that we are all supportive of eachother!\\

\section{Iteration X}

{\bf Date:} xxx\\
{\bf Start Time:} xxxx\\
{\bf End Time:} xxxx \\
{\bf Who:} xxx,xx,xxx (fullnames)\\
{\bf Where:} xxxx \\
{\bf Activities:} xxxxxxx\\
{\bf Outcomes:} xxxxx\\

%\section{Iteration 2}

%\section{Iteration 3}

\end{document}
